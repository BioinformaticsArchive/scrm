
\section{Blah Blah Introduction or something}

\section{Download and installation}
{\tt scrm} can be downloaded from \url{https://...}. Extract the source code by executing the following command:
\cm{tar -xf scrm-VERSION.tar.gz.}

It is fairly standard to compile {\tt scrm} on UNIX-like systems. In the directory of {\tt scrm-VERSION}, execute the following command:
\begin{verbatim}
$./bootstrap
$make
\end{verbatim}

\section{Basic command}
\cm{scrm nsam }


\cm{scrm nsam nreps -T [FILENAME]}


\section{Simulation for segregating sites???????}
\cm{scrm nsam nreps -t $\theta$}



\section{Recombination}


\cm{scrm nsam nreps -t $\theta$ -r $\rho$ [-nsitis NSITES] [-npop NPOP] }

\cm{scrm 6 3 -t 0.002 -r 0.00004 -npop 20000 }



\section{Options}
\begin{longtable}{lp{9cm}}
{\tt -T [FILENAME]} & Save output trees to file specified by {\tt FILENAME}. If {\tt FILENAME} is not given, trees are saved in file {\tt TREEFILE}.\\
 {\tt  -r} $\rho$ &  User define the recombination rate $\rho$, per gerneration per site.\\
 {\tt  -nsites} NSITES &  User define the sequence length NSITES. \\
  {\tt -t} $\theta$  &  User define the mutation rate THETA. \\
{\tt -npop} NPOP &  User define the population size NPOP. \\
 {\tt -seed} SEED  &  User define the random SEED. \\
{\tt -l} exact\_window\_length &  User define the length of the exact window. \\

 {\tt -log [LOGFILE]} & Save the log of the simulation to file specified by {\tt LOGFILE}. If {\tt LOGFILE} is not given, it is save at {\tt scrm.log} by default.  
\end{longtable}


\nocite{excoffier_fastsimcoal:_2011}

